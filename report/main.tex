\documentclass[12pt,a4paper]{report}
\usepackage{graphicx,epsfig}
\usepackage{amsfonts}
\usepackage{amsthm, amsmath}
\usepackage{xcolor}
\usepackage{textcomp}
\usepackage{listings}
\usepackage[authoryear,round]{natbib}
\usepackage[hidelinks]{hyperref}
\usepackage{todonotes}
\usepackage{mathtools}
\usepackage{algorithm2e}
\usepackage{graphicx}
\usepackage{svg}

\newcommand*\diff{\mathop{}\!\mathrm{d}}
\DeclareMathOperator{\trace}{trace}
\DeclareMathOperator*{\argmax}{arg\,max}
\DeclareMathOperator*{\argmin}{arg\,min}

\graphicspath{ {../code/figures/} }

\parindent=25pt
\parskip 0.10in

\date{}
\begin{document}
\thispagestyle{empty}
\begin{center}
{\huge
Gradient-Free Optimal Postprocessing of MCMC Output

\bigskip
\bigskip

by
\bigskip
\bigskip

Artem Glebov
}
\end{center}
\vfill

\begin{center}
{\large
Department of Mathematics\\
King's College London\\
The Strand, London WC2R 2LS\\
United Kingdom\\
\medskip

}
\end{center}
\bigskip


\newpage
\setcounter{page}{1}

\chapter*{Abstract}


\tableofcontents


\chapter*{Introduction}

The Python code accompanying this report can be found at:
\begin{center}
\url{https://github.com/aglebov/gradient-free-mcmc-postprocessing}.
\end{center}

The implementation of the gradient-free kernel Stein thinning was contributed by the author directly to the \texttt{stein-thinning} Python library:
\begin{center}
\url{https://github.com/wilson-ye-chen/stein_thinning}.
\end{center}

\chapter{Background}

\section{Markov chain Monte Carlo}

Markov chain Monte Carlo (MCMC) are a popular class of algorithms for sampling from complex probability distributions.

The need to sample from a probability distribution arises in exploratory analysis as well as when analytical expressions are unavailable for quantities of interest, such as the modes or quantiles of the distribution, or for expectations with respect to the distribution, so a numerical simulation is used to obtain approximations instead. Such cases are frequent in Bayesian analysis, where the posterior density often has a complex structure with an analytically intractable normalising constant.

\todo[inline]{Describe alternatives: the inverse method, accept-reject and importance sampling}

\todo[inline]{Include a simple motivating example}

An MCMC algorithm proceeds by sequentially constructing a chain of samples\footnote{These are also sometimes called ``draws''. In this report, we follow the literature in using the term ``sample'' both for a single element in an MCMC chain and for all such elements taken together as a sample from the target distribution.} $x_1$, $x_2$, $\dots$,  where each sample is drawn from a transition distribution $Q$ conditional on the preceding value:
$$x_{n+1} \sim Q(x_{n+1}|x_n).$$
The distribution $Q$ is known as the transition kernel and is selected so that it is easy to sample from and to ensure asymptotic convergence to the target distribution $\Pi$:
$$x_n \xrightarrow[]{d} \Pi \quad\text{as}\quad n \to \infty.$$

Two classical variations of this technique are the Metropolis-Hastings and Gibbs algorithms.

\paragraph{Metropolis-Hastings algorithm.} The algorithm due to \cite{metropolisEquationStateCalculations1953} and \cite{hastingsMonteCarloSampling1970} uses an auxiliary distribution $q$ to sample a proposed value
$$x' \sim q(x' | x_n),$$
which is then accepted with probability
$$\alpha(x_n, x') = 1 \wedge \frac{\pi(x')}{\pi(x_n)} \frac{q(x_n|x')}{q(x'|x_n)}.$$
If $x'$ is accepted, the algorithm sets $x_{n+1} = x'$. If $x'$ is rejected, the value remains unchanged: $x_{n+1} = x_n$.

\todo[inline]{Consider using a different notation to avoid the confusion between the density of the proposal $q$ and the transition kernel $Q$.}

The common choice for the proposal distribution $q$ is a symmetric proposal satisfying $q(x'|x_n) = q(x_n|x')$, so that the ratio of these two quantities disappears from the expression for the acceptance probability:
$$\alpha(x_n, x') = 1 \wedge \frac{\pi(x')}{\pi(x_n)}.$$
In the special case where $q(x'|x_n) = q(x' - x_n)$ we obtain a random walk proposal:
$$x' = x_n + Z,$$
where $Z$ is the distribution of the step taken by the algorithm, e.g. a multivariate normal distribution. The operation of the algorithm then resembles a random walk across the domain of the target distribution where steps towards areas of lower probability are more likely to be rejected. The scale of the step distribution $Z$ determines the average size of the jump that the algorithm can make in one iteration and thus the speed of traversal of the target domain.

An alternative to symmetric proposals is an independence proposal satisfying $q(x'|x_n) = q(x')$.

\todo[inline]{Cite the ST03 lecture notes or Robert \& Casella}

\paragraph{Gibbs algorithm.} Suppose $x$ is a $d$-dimensional vector and the components $x^{(1)}$, $x^{(2)}$, $\dots$, $x^{(d)}$ can be partitioned in such a way that we can sample the components belonging to each partition while keeping the components in other partitions fixed. That is, let $I_i \subset \{1, \dots, d\}$ with $\cup_{i=1}^k I_i = \{1, \dots, d\}$ for some $k$ and $I_i \cap I_j = \emptyset$ for $i \neq j$, and assume we can sample
$$x^{(I_i)} \sim f_i\left(x^{(I_i)} | x^{(I_1, \dots, I_{i-1}, I_{i+1}, \dots, I_k)}\right).$$
The sample $x_{n+1}$ can then be constructed by sequentially sampling for each partition:
$$x_{n+1}^{(I_i)} \sim f_i\left(x^{(I_i)} | x_{n+1}^{(I_1, \dots, I_{i-1})}, x_n^{(I_{i+1}, \dots, I_k)}\right).$$
Note that the newly sampled values $x_{n+1}^{(I_1, \dots, I_{i-1})}$ enter the computation for subsequent partitions.

\todo[inline]{Read and cite the original paper for Gibbs sampler}

\todo[inline]{Consider simplifying this description}

\todo[inline]{Mention HMC and other recent variations}

\section{Challenges of running MCMC}

While the asymptotic convergence of MCMC samples to the target distribution is guaranteed, no general guarantee is available for finite samples, resulting in several interrelated challenges that a practitioner faces when applying this class of algorithms:
\begin{enumerate}
\item The choice of a starting point for a chain affects the speed of convergence to the target distribution.
\item For a multimodal distribution, the algorithm might struggle to move between the modes within a feasible time. This problem becomes especially acute in high dimensions.
\item The scale of the proposal distribution must be calibrated to ensure that the algorithm is able to explore the domain of the target distribution efficiently.
\item Assessing how close an MCMC chain is to convergence is difficult, since the knowledge about the target distribution often comes from the chain itself.
\item In order to eliminate the impact of the starting point, it can be useful to discard the initial iterations of an MCMC chain, which are considered as ``burn-in''. Selecting the optimal length of the burn-in period is contingent on being able to detect convergence.
\item The sequential procedure of constructing a chain induces autocorrelation between the samples, which leads to increased variance of derived estimators.
\item The large number of samples resulting from an MCMC algorithm needs to be summarised for subsequent analysis, particularly when the cost of using all available samples is too high. Such situations arise when samples obtained from MCMC are used as starting points for further expensive simulations.
\end{enumerate}

The first three challenges require decisions to be made upfront before running the algorithm or adaptively during its run. In order to address the impact of the starting point, running multiple chains with starting points sampled from an overdispersed distribution is recommended~(\cite{gelmanInferenceIterativeSimulation1992}). This approach has the added benefit of increasing the chance of discovering the modes of the target distribution, although it does not provide a guarantee in this respect. 
\todo[inline]{Mention perfect sampling.}

Comparing the summary statistics of several chains (\cite{gelmanInferenceIterativeSimulation1992,brooksGeneralMethodsMonitoring1998,vehtariRankNormalizationFoldingLocalization2021}) offers a way to detect a lack of convergence at the cost of additional computation. Alternatively, the comparison can be applied to batches of samples from a single chain, as proposed by \cite{vatsRevisitingGelmanRubin2021}. Convergence detection can be used to terminate the algorithm once a chosen criterion is satisfied, or to assess the quality of a sample retrospectively. It should be noted that convergence criteria establish a necessary but not sufficient condition for convergence, so the outcomes need to be interpreted accordingly.

The scaling of the step distribution in a random-walk Metropolis-Hastings algorithm is commonly tuned to target the acceptance rate of roughly 0.234 for proposed samples (\cite{gelmanEfficientMetropolisJumping1996,gelmanWeakConvergenceOptimal1997,robertsOptimalScalingVarious2001}), which balances the speed of traversal and the computational effort generating samples that end up rejected.

The last three challenges are typically addressed by post-processing a sample from a completed MCMC run. A recent proposal by~\cite{riabizOptimalThinningMCMC2022} addresses these challenges by selecting a fixed-size subset of samples from an MCMC run such that the empirical distribution given by the subset best approximates the distribution resulting from the full sample. In the following section, we consider their approach in greater detail.

\todo[inline]{Read and cite Cowles and Carlin (1996) regarding the choice of burn-in length.}

\section{Optimal thinning}

Given a Markov chain $(X_i)_{i \in \mathbb{N}}$ and its realisation $(x_i)_{i=1}^n$ of length $n$, the empirical approximation of the target posterior distribution is given by
\begin{equation}
\frac{1}{n} \sum_{i=1}^n \delta(x_i),
\label{eq:discrete-distribution}
\end{equation}
where $\delta(x)$ is the Dirac delta function.
\cite{riabizOptimalThinningMCMC2022} set out to identify $m \ll n$ indices $\pi(j) \in \{1,\dots, n\}$ with $j\in\{1, \dots, m\}$, such that the approximation provided by the subset of samples
\begin{equation}
\frac{1}{m} \sum_{j=1}^m \delta(x_{\pi(j)})
\label{eq:thinned-sample}
\end{equation}
is closest to the approximation given by the full set, and thus to the target distribution. The criterion of proximity is based on the kernel Stein discrepancy, itself a special case of the integral probability metric.

\subsection{Kernel Stein discrepancy}

The integral probability metric between two distributions $P$ and $P'$ is defined as
\begin{equation}
\mathcal{D}_{\mathcal{F}}(P, P') \coloneq \sup_{f \in \mathcal{F}}\left|\int_\mathcal{X} f \diff P - \int_\mathcal{X} f \diff P' \right|,
\label{eq:ipm}
\end{equation}
where $\mathcal{X}$ is a measurable space on which both $P$ and $P'$ are defined and $\mathcal{F}$ is a set of test functions. Depending on the choice of $\mathcal{F}$, the definition~(\ref{eq:ipm}) gives rise to different classes of probability metrics, including the well-known Kolmogorov distance, Wasserstein distance and total variation distance.

If $P$ is taken to be the target distribution of an MCMC algorithm, evaluating~(\ref{eq:ipm}) poses two practical challenges:
\begin{itemize}
\item the integral $\int_\mathcal{X} f \diff P$ is often analytically intractable,
\item the supremum requires a non-trivial optimisation procedure to find.
\end{itemize}

The need to integrate with respect to $P$ can be eliminated if we find a set of function $\mathcal{F}$ for which $\int_\mathcal{X} f \diff P = 0$ for all $f \in \mathcal{F}$. The expression~(\ref{eq:ipm}) then simplifies to
\begin{equation}
\mathcal{D}_{\mathcal{F}}(P, P') = \sup_{f \in \mathcal{F}}\left|\int_\mathcal{X} f \diff P' \right|.
\label{eq:stein-discrepancy-sup}
\end{equation}

\cite{gorhamMeasuringSampleQuality2015} propose choosing such a set $\mathcal{F}$ based on the observation that the infinitesimal generator
$$(\mathcal{L}u)(x) = \lim_{t \to 0} \frac{\mathbb{E}[u(Z_t) | Z_0 = x] - u(x)}{t} \quad \text{for } u:\mathbb{R}^d \to \mathbb{R}$$
of a Markov process $(Z_t)_{t \geq 0}$ with stationary distribution $P$ satisfies 
$$\mathbb{E}[(\mathcal{L} u)(Z)] = 0$$
under mild conditions on $\mathcal{L}$ and $u$\todo[inline]{Check the conditions.}. In the specific case of an overdamped Langevin diffusion
$$\diff Z_t = \frac{1}{2} \nabla \log p(Z_t) \diff t + \diff W_t,$$
where $W_t$ is the standard Brownian motion, the infinitesimal generator becomes
$$(\mathcal{L}_P u)(x) = \frac{1}{2} \langle \nabla u(x), \nabla \log p(x)\rangle + \frac{1}{2}\langle \nabla, \nabla u(x) \rangle.$$
Denoting $g  = \frac{1}{2}\nabla u$, they obtain the Stein operator
\begin{equation}
\mathcal{A}_P g = \langle g, \nabla \log p \rangle + \langle \nabla, g \rangle = \langle p^{-1}\nabla, p g \rangle,
\label{eq:stein-operator}
\end{equation}
and rewrite~(\ref{eq:stein-discrepancy-sup}) as
\begin{equation}
\mathcal{D}_{P, \mathcal{G}}(P') = \sup_{g \in \mathcal{G}}\left|\int_\mathcal{X} \mathcal{A}_P g \diff P' \right|
\label{eq:stein-discrepancy-g}
\end{equation}
for a suitably chosen set $\mathcal{G}$. Note that $p$ enters~(\ref{eq:stein-operator}) via $\nabla \log p$, so the knowledge of its normalising constant is not required to evaluate the operator.

To remove the optimisation step in the calculation of~(\ref{eq:stein-discrepancy-g}), \cite{gorhamMeasuringSampleQuality2017} employ a reproducing kernel Hilbert space (RKHS) $\mathcal{H}(k)$ with kernel $k: \mathbb{R}^d \times \mathbb{R}^d \to \mathbb{R}$ satisfying the reproducing property:
$$f(x) = \langle f, k(x, \cdot)\rangle \quad\text{for } f \in \mathcal{H}(k).$$
Taking the set
\begin{equation}
\mathcal{G} \coloneq \left\{ \mathrm{g} : \mathbb{R}^d \to \mathbb{R}^d \left| \sum_{i=1}^d \|g_i\|^2_{\mathcal{H}(k)} \leq 1 \right.\right\}
\label{eq:unit-ball}
\end{equation}
which defines a unit-ball in a Cartesian product of $d$ copies $\mathcal{H}(k)$, Proposition~2 in \cite{gorhamMeasuringSampleQuality2017} establishes that
\begin{equation}
\mathcal{D}_{P}^2(P') = \mathbb{E}_{P' \times P'}[k_P(X, \tilde{X})] = \iint_\mathcal{X} k_P(x, y) \diff p'(x) \diff p'(y),
\label{eq:stein-discrepancy-sqrt-expectation}
\end{equation}
where $X, \tilde{X} \sim P'$, $p'$ is the density of $P'$, and $k_P(x, y)$ is given by
\begin{equation}
\begin{aligned}
k_P(x, y) \coloneq 
&(\nabla_x\cdot\nabla_y) k(x,y) \\
&+ \langle \nabla_x k(x, y), \nabla_y \log p(y) \rangle + \langle \nabla_y k(x, y), \nabla_x \log p(x) \rangle \\
&+ k(x, y) \langle \nabla_x \log p(x), \nabla_y \log p(y) \rangle.
\label{eq:deriv:stein-kernel}
\end{aligned}
\end{equation}
Here $\nabla_x$ and $\nabla_y$ are gradients w.r.t.\ $x$ and $y$, respectively, and the operator $\nabla_x\cdot\nabla_y$ is defined as:
$$(\nabla_x\cdot\nabla_y) k(x,y) \coloneq \sum_{i=1}^d \frac{\partial^2}{\partial x_i\, \partial y_i} k(x, y).$$
Note that we have dropped $\mathcal{G}$ in the subscript of~(\ref{eq:stein-discrepancy-sqrt-expectation}) as the choice of $\mathcal{G}$ does not depend on $P$ or $P'$.
If $P'$ is a discrete distribution, as in~(\ref{eq:discrete-distribution}), the squared discrepancy~(\ref{eq:stein-discrepancy-sqrt-expectation}) becomes
\begin{equation}
\mathcal{D}_{P}^2\left(\frac{1}{n} \sum_{i=1}^n \delta(x_i)\right) = \frac{1}{n^2} \sum_{i,j=1}^n k_P(x_i, x_j).
\label{eq:ksd:discrete}
\end{equation}

When $k(x, y)$ is chosen to be the inverse multiquadratic kernel (IMQ)
\begin{equation}
k(x, y) = \left(c^2 + \|\Gamma^{-1/2}(x-y)\|\right)^\beta
\label{eq:imq}
\end{equation}
with $\beta \in (-1, 0)$, \cite{gorhamMeasuringSampleQuality2017} demonstrate for $\Gamma = I$ that $\mathcal{D}_{P}(P')$ provides convergence control:
\begin{itemize}
\item if $\mathcal{D}_{P}(P'_m) \to 0$, then $P'_m$ converges in distribution to $P$ (Theorem 8),
\item if $P'_m$ converges to $P$ in Wasserstein distance, then $\mathcal{D}_{P}(P'_m) \to 0$ (Proposition 9)
\end{itemize}
for a sequence of distributions $P'_m$ under suitable conditions. Theorem 4 by \cite{chenSteinPointMarkov2019} justifies the introduction of $\Gamma$ in IMQ.

The constant $c$ in~(\ref{eq:imq}) can be set to 1 without loss of generality, and the positive-definite preconditioner matrix $\Gamma$ can be chosen to exploit the geometry of the parameter space. \cite{riabizOptimalThinningMCMC2022} suggest several choices for $\Gamma$, in particular the identity matrix scaled by the median Euclidean distance between points in the sample, possibly further rescaled by $(\log m)^{-1/2}$ where $m$ is the desired cardinality of the thinned sample, or the sample covariance matrix. Based on empirical evaluation, \cite{gorhamMeasuringSampleQuality2017} and \cite{riabizOptimalThinningMCMC2022} settle on the value $\beta = -\frac12$.

The expression for $k_P(x, y)$ when $k(x, y)$ is taken to be the inverse multiquadratic kernel is derived in section~\ref{appendix:derivations:imq-stein} and is coded directly in the Python library \texttt{stein-thinning}\footnote{Available from \url{https://github.com/wilson-ye-chen/stein_thinning}.}.

Other choices of kernels are possible and offer convergence control, as demonstrated by \cite{chenSteinPoints2018}, however IMQ performed on par or better than the alternatives considered by the authors.

\subsection{Stein thinning}
\label{sec:stein-thinning}

Equipped with the kernel Stein discrepancy (KSD) as defined above, \cite{riabizOptimalThinningMCMC2022} develop a greedy optimisation algorithm to select a subset of points from a sample that minimises the total kernel Stein discrepancy. Rather than attempting to evaluate KSD for all $n \choose m$ combinations of points, they construct a subsample iteratively, each time picking a point that minimises the KSD with previously selected points. We reproduce their procedure verbatim in Algorithm~\ref{alg:cap} for the reader's convenience. 

\begin{algorithm}
\caption{Stein thinning.}\label{alg:cap}
\KwData{Sample $(x_i)_{i=1}^n$ from MCMC, Stein kernel $k_P$, desired cardinality $m \in \mathbb{N}$.}
\KwResult{Indices $\pi$ of a sequence $(x_{\pi(j)})_{j=1}^m \subset \{x_i\}_{i=1}^n$ where $\pi(j) \in \{1, \dots, n\}$.}

\For{$j = 1, \dots, m$}{
$$\pi(j) \in \argmin_{i=1,\dots,n} \frac{k_P(x_i, x_i)}{2} + \sum_{j'=1}^{j-1} k_P(x_{\pi(j'}, x_i)$$
}
\end{algorithm}

The algorithm was implemented by the authors and made available in the open-source library \texttt{stein-thinning}. The computational complexity of the provided implementation is $O(nm)$.

The strength of the method lies in its ability to correct for bias in the input sample, as established by Theorem 3 in \cite{riabizOptimalThinningMCMC2022}, meaning that the algorithm can be applied to samples from MCMC chains that have not converged to the target distribution, provided that its domain is sufficiently explored by the chains. The limitation of the method comes from its reliance on the gradients of the log-target, which may be expensive to compute.

\subsection{Gradient-free kernel Stein discrepancy}

\todo[inline]{Cite cases where gradient cannot be computed easily.}

To address the limitation of the kernel Stein discrepancy, \cite{fisherGradientFreeKernelStein2024} propose the gradient-free Stein operator $\mathcal{S}_{P,Q}$ defined for any differentiable function $g$ as
\begin{equation}
\mathcal{S}_{P,Q} g \coloneq \frac{q}{p}(\langle g, \nabla \log q \rangle + \langle \nabla, g \rangle) = \frac{q}{p} \langle q^{-1}\nabla, q g \rangle.
\end{equation}
This definition generalises expression~(\ref{eq:stein-operator}) by introducing a proxy distribution $Q$ with density $q$ chosen such that its gradient is easily computable. When $q = p$, we recover the original Langevin Stein operator~(\ref{eq:stein-operator}).

\cite{fisherGradientFreeKernelStein2024} proceed to show in their Proposition 1 that, under certain regularity conditions,
\begin{equation*}
\int_\mathcal{X} \mathcal{S}_{P,Q} g \diff P = 0
\end{equation*}
for any function $g$ whose first derivatives exist and are bounded, allowing them to define the gradient-free kernel Stein discrepancy
\begin{equation*}
\mathcal{D}_{P, Q}(P') = \sup_{g \in \mathcal{G}}\left|\int_\mathcal{X} \mathcal{S}_{P,Q} g \diff P' \right|
\end{equation*}
by analogy with~(\ref{eq:stein-discrepancy-g}), taking $\mathcal{G}$ again to be the unit-ball~(\ref{eq:unit-ball}). Futhermore, Proposition 7 in \cite{fisherGradientFreeKernelStein2024} establishes that
\begin{equation*}
\mathcal{D}_{P, Q}^2(P') = \iint_\mathcal{X} \frac{q(x)}{p(x)} \frac{q(y)}{p(y)} k_Q(x, y) \diff p'(x) \diff p'(y),
\label{eq:gf-ksd:int}
\end{equation*}
where $p'$ is the density of distribution $P'$ and $k_Q(x, y)$ is given by~(\ref{eq:deriv:stein-kernel}) but with $q$ replacing $p$. For a discrete $P'$, this translates to
\begin{equation}
\mathcal{D}_{P, Q}^2\left(\frac{1}{n} \sum_{i=1}^n \delta(x_i)\right) = \frac{1}{n^2} \sum_{i,j=1}^n \frac{q(x_i)}{p(x_i)} \frac{q(x_j)}{p(x_j)} k_Q(x_i, x_j).
\label{eq:gf-ksd:discrete}
\end{equation}

Both expression~(\ref{eq:ksd:discrete}) and~(\ref{eq:gf-ksd:discrete}) can be viewed as averaging the elements of matrices, and this fact is used to provide an efficient implementation of the greedy search described in Section~\ref{sec:stein-thinning}.

When $k_Q(x,y)$ is based on IMQ with $\Gamma = I$, Theorem 2 in \cite{fisherGradientFreeKernelStein2024} proves that $\mathcal{D}_{P, Q}^2(P_m') \to 0$ implies that the sequence of distributions $P'_m$ converges weakly to $P$ as $m \to \infty$, thus ensuring convergence control. 

While removing the need to calculate the gradient of log-target, the approach by \cite{fisherGradientFreeKernelStein2024} replaces it with the requirement to choose a suitable proxy distribution $Q$. The authors provide several examples:
\begin{itemize}
\item where the target distribution is the posterior in a Bayesian inference problem, the prior distribution can serve as $Q$,
\item where $p$ can be differentiated, the Laplace approximation of $P$ could be used,
\item where samples from $P$ are available, it can be approximated by either the Gaussian mixture model (GMM) or a kernel density estimator (KDE).
\end{itemize}

The choice of a proxy is non-trivial, however, and \cite{fisherGradientFreeKernelStein2024} warn of a possible failure of convergence control when the proxy density $q$ has either a substantially heavier or a substantially lighter tail than $p$. Two other situations are identified by the authors as detrimental to their approach: high dimension of the domain $\mathcal{X}$ and well separated high-probability regions.

Two applications of gradient-free KSD are considered by \cite{fisherGradientFreeKernelStein2024}: importance resampling and variational inference. In this dissertation, we adopt their approach for the thinning problem.

\chapter{Methodology and Data}

We modify Algorithm~\ref{alg:cap} to use the gradient-free Stein kernel $k_{P,Q}(x,y)$ in place of $k_P(x, y)$. With inverse multiquadratic kernel~(\ref{eq:imq}), this results in the expression
\begin{equation}
\begin{aligned}
k_{P,Q}(x, y)
= &\frac{q(x)}{p(x)} \frac{q(y)}{p(y)} \times \\
 &\left[ -4 \frac{\beta(\beta-1) \| \Gamma^{-1}(x - y)\|^2}{(c^2 + \| \Gamma^{-1/2}(x-y)\|^2)^{-\beta-2}} \right.  \\
&- 2 \beta \frac{\trace(\Gamma^{-1}) + \langle \Gamma^{-1} (x - y), \nabla_x \log q(x) - \nabla_y \log q(y)\rangle}{(c^2 + \| \Gamma^{-1/2}(x-y)\|^2)^{-\beta-1}} \\
& \left. + \frac{\langle \nabla_x \log q(x), \nabla_y \log q(y) \rangle}{(c^2 + \| \Gamma^{-1/2}(x-y)\|^2)^{-\beta}} \right]. \\
\end{aligned}
\label{eq:k_Q}
\end{equation}

We aim to compare the performance of the proposed algorithm against na\"ive thinning (retaining each $i$-th element of the sample) and the Stein thinning algorithm of \cite{riabizOptimalThinningMCMC2022} for several target distributions. The outline of the evaluation procedure for each test case is as follows:
\begin{enumerate}
\item obtain a sample from the target distribution,
\item apply na\"ive thinning, Stein thinning and the proposed algorithm to get a thinned sample of a given cardinality,
\item calculate the energy distance between the thinned sample and the target distribution and compare the values between the algorithms.
\end{enumerate}

\section{Test Cases}
\label{sec:methodology:data}

We use synthetic data in order to have control over the ground truth in our experiments.

\subsection{Bivariate Gaussian mixture}

The purpose of the first test case is to confirm the expected behaviour of the proposed algorithm in a simple setting. Here we take the target distribution to be the bivariate Gaussian mixture with means 
$$
\mu_1 = \begin{pmatrix} -1 \\ -1 \end{pmatrix}, \qquad
\mu_2 = \begin{pmatrix} 1 \\ 1 \end{pmatrix},
$$
covariance matrices
$$
\Sigma_1 = \begin{pmatrix}
0.5 & 0.25\\
0.25 & 1
\end{pmatrix}, \qquad
\Sigma_2 = \begin{pmatrix}
2 & -0.8 \sqrt{3}\\
-0.8 \sqrt{3} & 1.5
\end{pmatrix}
$$
and weights
$$w = \begin{pmatrix} 0.3 \\ 0.7 \end{pmatrix}.$$

We obtain 1000 samples by directly drawing from the target. A scatter plot of the sample as well as the contour plot of the probability density are shown in Figure~\ref{fig:gmm:sample}.

\begin{figure}[h]
\centering
\makebox[\textwidth][c]{
	\includesvg[width=1.0\textwidth]{gaussian-mixture-sample}
}
\caption{(a) Sample from the bivariate Gaussian mixture with two components and (b) its probability density function.
\label{fig:gmm:sample}}
\end{figure}

\subsection{Lotka-Volterra inverse problem}

The Lotka-Volterra model describes the evolution of an idealised ecosystem with two species: predator and prey. The predator population grows when prey is abundant, and the prey population shrinks when there are too many predators. Denoting the size of prey population by $u_1$, and the predator population by $u_2$, the model postulates the following dynamic:
%\begin{equation}
%\begin{alignat*}{4}
%\frac{\diff u_1}{u_1} & = ( &  \theta_1 & & - \theta_2 u_2 & ) & & \diff t \\
%\frac{\diff u_2}{u_2} & = ( & -\theta_3 & & + \theta_4 u_1 & ) & & \diff t \\
%\end{alignat*}
%\end{equation}
\begin{equation}
\begin{aligned}
\frac{\diff u_1}{u_1} & = ( \;\;\;\theta_1 - \theta_2 u_2 ) \diff t \\
\frac{\diff u_2}{u_2} & = ( -\theta_3 + \theta_4 u_1 ) \diff t \\
\end{aligned}
\label{eq:lotka-volterra}
\end{equation}
The resulting behaviour is driven by the four parameters $\theta_1, \dots, \theta_4$.

If a realisation of $u_1(t)$ and $u_2(t)$ is observed for an interval $[t_1, t_2]$, one may wish to infer the values of the parameters that best describe the observed behaviour. In a practical setting, this corresponds to formulating a parametric model for a natural phenomenon and inferring the parameters of the model from the measurements of the phenomenon. We emulate this situation by fixing the true model, generating a realisation perturbed by noise and then attempting to recover the parameters of the true model from the realisation.

We take $u_1(t)$ and $u_2(t)$ to be generated for $t \in [0, 25]$ by the model~(\ref{eq:lotka-volterra}) with parameters $(0.67, 1.33, 1, 1)$ and initial values $u_1(0) = 1$, $u_2(0) = 1$. The interval $[0, 25]$ is discretised into $2400$ points. Gaussian noise with mean 0 and standard deviation 0.2 is then added to all data points $u_1(t)$ and $u_2(t)$. For comparability of results, we use the same values as those used by \cite{riabizOptimalThinningMCMC2022}. Figure~\ref{fig:lotka-volterra:data} displays the resulting dataset.

\begin{figure}[h]
\centering
\makebox[\textwidth][c]{
	\includesvg[width=1.0\textwidth]{lotka-volterra}
}
\caption{Solution to the Lotka-Volterra ODEs (black) with added Gaussian noise (grey).
\label{fig:lotka-volterra:data}}
\end{figure}


\section{Evaluating the approximation}
\label{sec:methodology:evaluation}

In order to assess how well the selected sample approximates the posterior distribution, we use the energy distance. Following \cite{rizzoEnergyDistance2016}, the squared energy distance is defined for two distributions $F$ and $G$ as
$$D^2(F, G) \coloneq 2 \mathbb{E} \|X - Y\| - \mathbb{E}\|X - X'\| - \mathbb{E} \|Y - Y'\|,$$
where $X, X' \sim F$, $Y, Y' \sim G$, and $\|\cdot\|$ denotes the Euclidean norm. For samples $x_1, \dots, x_n$ and $y_1, \dots, y_m$ from $X$ and $Y$, respectively, the corresponding statistic is given by
\begin{equation}
\mathcal{E}_{n,m}(X, Y) \coloneq \frac{2}{nm}\sum_{i=1}^n \sum_{j=1}^m \|x_i - y_j\| - \frac{1}{n^2} \sum_{i=1}^n\sum_{j=1}^n \|x_i - x_j\| - \frac{1}{m^2} \sum_{i=1}^m \sum_{j=1}^m \|y_i - y_j\|.
\label{eq:energy-distance:discrete}
\end{equation}

\todo[inline]{Advantages: objective and easy to calculate.}

\chapter{Results}

\chapter{Conclusions}

\appendix
\chapter{Derivations}
\label{appendix:derivations}

\section{Stein kernel based on inverse multiquadratic kernel}
\label{appendix:derivations:imq-stein}

Given a kernel $k(x,y)$, the corresponding Stein kernel is given by~(\ref{eq:deriv:stein-kernel}). For the inverse multiquadratic kernel~(\ref{eq:imq}) we obtain
\begin{equation}
\begin{aligned}
\frac{\partial}{\partial x_r} k(x,y) 
%&= \beta \left(c^2 + \sum_{i=1}^d\sum_{j=1}^d (x_i-y_i) \Gamma^{-1}_{ij}(x_j-y_j)\right)^{\beta-1} \\
%&\times \left( \sum_{j=1}^d \Gamma^{-1}_{lj}(x_j - y_j) + \sum_{i=1}^d (x_i - x_j) \Gamma^{-1}_{il} \right) \\
&= \beta \left(c^2 + \sum_{i=1}^d\sum_{j=1}^d (x_i-y_i) \Gamma^{-1}_{ij}(x_j-y_j)\right)^{\beta-1} \\
&\times \sum_{j=1}^d (\Gamma^{-1} + \Gamma^{-T})_{rj}(x_j - y_j) \\
&= 2 \beta \left(c^2 + \sum_{i=1}^d\sum_{j=1}^d (x_i-y_i) \Gamma^{-1}_{ij}(x_j-y_j)\right)^{\beta-1}
\sum_{j=1}^d \Gamma^{-1}_{rj}(x_j - y_j), \\
\end{aligned}
\end{equation}
where we used that $\Gamma$ is a symmetric matrix. The gradient is then
\begin{equation}
\nabla_x k(x,y) = 2 \beta \left(c^2 + \| \Gamma^{-1/2} (x-y)\|^2\right)^{\beta-1} \Gamma^{-1} (x - y).
\label{eq:appx:deviv:nablax}
\end{equation}
and similarly
\begin{equation}
\nabla_y k(x,y) = -2 \beta \left(c^2 + \| \Gamma^{-1/2} (x-y)\|^2\right)^{\beta-1} \Gamma^{-1} (x - y).
\label{eq:appx:deviv:nablay}
\end{equation}
Now
\begin{equation}
\begin{aligned}
\frac{\partial^2}{\partial x_r\,\partial y_r} k(x,y) 
&= -4 \beta(\beta-1) \left(c^2 + \sum_{i=1}^d\sum_{j=1}^d (x_i-y_i) \Gamma^{-1}_{ij}(x_j-y_j)\right)^{\beta-2} \\
&\times \left(\sum_{j=1}^d \Gamma^{-1}_{rj}(x_j - y_j)\right)^2 \\
&- 2\beta \left(c^2 + \sum_{i=1}^d\sum_{j=1}^d (x_i-y_i) \Gamma^{-1}_{ij}(x_j-y_j)\right)^{\beta-1} \Gamma^{-1}_{rr}
\end{aligned}
\end{equation}
which gives us
\begin{equation}
\begin{aligned}
(\nabla_x \cdot \nabla_y) k(x,y) 
&= -4 \beta(\beta-1) \left(c^2 + \| \Gamma^{-1/2}(x-y)\|^2\right)^{\beta-2} \| \Gamma^{-1}(x - y)\|^2 \\
&- 2\beta \left(c^2 + \|\Gamma^{-1/2}(x-y)\|^2\right)^{\beta-1} \trace(\Gamma^{-1})
\label{eq:appx:deriv:nablax_nablay}
\end{aligned}
\end{equation}

Substituting (\ref{eq:appx:deviv:nablax}), (\ref{eq:appx:deviv:nablay}) and (\ref{eq:appx:deriv:nablax_nablay}) into (\ref{eq:deriv:stein-kernel}), we obtain
\begin{equation}
\begin{aligned}
k_P(x, y)
= &-4 \beta(\beta-1) \left(c^2 + \| \Gamma^{-1/2}(x-y)\|^2\right)^{\beta-2} \| \Gamma^{-1}(x - y)\|^2 \\
&- 2\beta \left(c^2 + \|\Gamma^{-1/2}(x-y)\|^2\right)^{\beta-1} \trace(\Gamma^{-1}) \\
&+ 2 \beta \left(c^2 + \| \Gamma^{-1/2} (x-y)\|^2\right)^{\beta-1} \langle \Gamma^{-1} (x - y), \nabla_y \log p(y)\rangle \\
&- 2 \beta \left(c^2 + \| \Gamma^{-1/2} (x-y)\|^2\right)^{\beta-1} \langle \Gamma^{-1} (x - y), \nabla_x \log p(x)\rangle \\
&+ \left(c^2 + \| \Gamma^{-1/2} (x-y)\|^2\right)^\beta \langle \nabla_x \log p(x), \nabla_y \log p(y) \rangle \\
= &-4 \beta(\beta-1) D^{\beta-2} \| \Gamma^{-1}(x - y)\|^2  \\
&- 2 \beta D^{\beta-1} (\trace(\Gamma^{-1}) + \langle \Gamma^{-1} (x - y), \nabla_x \log p(x) - \nabla_y \log p(y)\rangle) \\
&+ D^\beta \langle \nabla_x \log p(x), \nabla_y \log p(y) \rangle, \\
\end{aligned}
\label{eq:k_P:IMQ}
\end{equation}
where we have denoted $D = c^2 + \| \Gamma^{-1/2}(x-y)\|^2$.


\bibliographystyle{plainnat_modified}
\bibliography{biblio}

\end{document}
