\documentclass[12pt,a4paper]{report}
\usepackage{graphicx,epsfig}
\usepackage{amsfonts}
\usepackage{amsthm, amsmath}
\usepackage{xcolor}
\usepackage{textcomp}
\usepackage{listings}
\usepackage[authoryear,round]{natbib}

\parindent=25pt
\parskip 0.10in

% The following commands make it easy to include matlab code
% into your dissertation


\date{}
\begin{document}
\thispagestyle{empty}
\begin{center}
{\huge
Gradient-Free Optimal Postprocessing of MCMC Output

\bigskip
\bigskip

by
\bigskip
\bigskip

Artem Glebov
}
\end{center}
\vfill

\begin{center}
{\large
Department of Mathematics\\
King's College London\\
The Strand, London WC2R 2LS\\
United Kingdom\\
\medskip

}
\end{center}
\bigskip


\newpage
\setcounter{page}{1}

\chapter*{Abstract}
Some Abstract

\tableofcontents


\chapter*{Introduction}
Introduction to your project.

\chapter{Background and data}
One or more chapters presenting the necessary background for the methods you are going to use and discuss and the data you are going to analyse. It should include relevant references, like this: \cite{R}.

\chapter{Methodology}
One or more chapters describing the novel methodology you have developed or implemented or the strategy for model comparisons and assessment. 


\chapter{Results}
The results of your analysis.

\chapter{Conclusions}
What the reader has learnt from your dissertation and what questions are still open.  
\appendix
\chapter{Code}
\label{appendix:code}
 Here you can include relevant bits of the code.


\bibliographystyle{plainnat}
\bibliography{biblio}

\end{document}
